\documentclass[12pt]{article}
\usepackage[a4paper,margin=2.5cm]{geometry}
\usepackage{amsmath,amssymb}
\usepackage{enumitem}
\usepackage{hyperref}

\title{Test yourself questions}
\author{FYS-STK3155/4155}
\date{Last weekly exercise set}

\begin{document}
\maketitle


\section{Linear Regression}

\begin{enumerate}[leftmargin=1.2cm]

\item[\textbf{1.}]\textbf{(Multiple Choice)}  
Which of the following is \emph{not} an assumption of ordinary least squares linear regression?  
\begin{enumerate}[label=\alph*)]
\item Linearity between predictors and target  
\item Normality of predictors/features  
\end{enumerate}
Linearity: True. The relationship between predictors and the outcome is linear.
Normality of predictors: Each independent feature is normally distributed.
This is false, linear regression 
does not require the predictors themselves to be normally distributed.

\item[\textbf{2.}]\textbf{(True/False)}  
The mean squared error cost function for linear regression is convex in the parameters, guaranteeing a unique global minimum.
Answer: The MSE cost in linear regression is a convex quadratic function, so gradient-based optimization will find the global minimum .


\end{enumerate}

%%%%%%%%%%%%%%%%%%%%%%%%%%%%%%%%%%%%%%%%%%%%%%%%%%%%
\section{Logistic Regression}

\begin{enumerate}[leftmargin=1.2cm,start=5]

\item[\textbf{5.}]\textbf{(Multiple Choice)}  
Which statement about logistic regression is \emph{false}?  
\begin{enumerate}[label=\alph*)]
\item Used for binary classification  
\item Uses sigmoid to map linear scores to probabilities  
\item Has an analytical closed-form solution  
\item The log-loss is convex  
\end{enumerate}
1. True. Logistic regression is used for binary classification problems (outputs a probability for the positive class).


2. True It uses the logistic (sigmoid) function to map linear combinations of features to probabilities.


3. True. It has an analytical closed-form solution for its parameters, analogous to the normal equation in linear regression, however it needs to be solved numerically.


4. Its loss function (log loss) is convex, which guarantees a unique global optimum during training.
Logistic regression is a probabilistic classifier for binary outcomes, learned via maximum likelihood. Unlike linear regression, it does not have a closed-form coefficient solver and must be fit with iterative methods. Its negative log-likelihood (cross-entropy) cost is convex, ensuring a single global minimum .

\item[\textbf{6.}]\textbf{(True/False)}  
Logistic regression produces a linear decision boundary in the input feature space.
Answer: True. Logistic regression (with no feature transformations) produces a linear decision boundary in the input feature space.
True. The model is $\sigma(w^T x + b)$ and the decision boundary occurs at $w^T x + b = 0$, which is a hyperplane, that is a linear boundary.
\item[\textbf{7.}]\textbf{(Short Answer)}  

Give two reasons why logistic regression is preferred over linear
regression for binary classification.  Answer: First, logistic
regression outputs probabilities (via the sigmoid function), which are
naturally bounded between 0 and 1, whereas linear regression can
produce arbitrary values not suited for classification.  Second, using
linear regression for classification (with a threshold) can be
problematic since it treats errors on 0/1 outcomes in a least-squares
sense, which can lead to unstable or nonsensical thresholds and is not
theoretically well-founded. Logistic regression instead optimizes a
log-loss (cross-entropy) cost, which is well-suited for binary
outcomes and often yields better calibrated probability
estimates. Additionally, logistic regression is less sensitive to
class imbalance than linear regression with a threshold, and its loss
function is convex, avoiding some of the issues linear regression
would have on classification tasks.

\end{enumerate}

%%%%%%%%%%%%%%%%%%%%%%%%%%%%%%%%%%%%%%%%%%%%%%%%%%%%
\section{Neural Networks (Feedforward)}

\begin{enumerate}[leftmargin=1.2cm,start=9]

\item[\textbf{9.}]\textbf{(Multiple Choice)}  
Which statement is \emph{not} true for fully-connected neural networks?  
\begin{enumerate}[label=\alph*)]
\item Without nonlinearities they reduce to a single linear model  
\item Backpropagation applies the chain rule  
\item A single hidden layer can approximate any continuous function  
\item The loss surface is convex  
\end{enumerate}
1. Without non-linear activation functions in the hidden layers, the network would be equivalent to a single linear model (no matter how many layers are stacked).


2. True. Training deep neural networks relies on backpropagation, which uses the chain rule of calculus to compute gradients for all weights in the network.


3. True. With enough hidden units, a neural network with even a single hidden layer can approximate any continuous function on a closed interval (given mild conditions on the activation function).


4. False. The loss surface of a deep neural network (with two or more hidden layers) is convex, so any local minimum of the training loss is also a global minimum.
(Neural networks require non-linear activations to gain expressive power; otherwise multiple layers collapse to an equivalent single linear transformation . They are universal approximators in theory and are trained via backpropagation (chain rule for gradients). However, the loss surface for deep nets is non-convex, generally possessing many local minima and saddle points.

\item[\textbf{10.}]\textbf{(True/False)}  
Using sigmoid activations in deep networks can cause vanishing gradients.
True. Sigmoid/tanh activations squash outputs to (0,1)/(−1,1), and their derivatives are small for large magnitude inputs. In a deep network, gradients propagated backward can thus diminish exponentially, “vanishing” before reaching early layers. This makes it difficult for those layers to learn.
\item[\textbf{11.}]\textbf{(Short Answer)}  
The vanishing gradient problem refers to the tendency of gradients to become extremely small in early layers of a deep network during training. It occurs because gradients are the product of many small partial derivatives from the chain rule.  In deep networks (especially with sigmoid or $\tanh$ activations), these derivatives can be less than 1, causing the product to shrink exponentially as it is backpropagated through many layers. As a result, the early (lower) layers learn very slowly since their weights receive almost no update. A common technique to mitigate this is to use ReLU (Rectified Linear Unit) activations (or other activation functions that do not saturate) in place of sigmoids. ReLUs have derivative 0 or 1, which helps maintain larger gradients. Other strategies include careful weight initialization, batch normalization, residual connections, or using architectures like LSTMs (for RNNs) that are designed to preserve gradients.

\item[\textbf{12.}]\textbf{(Short Answer)}  
Given layer sizes $n_0,n_1,\dots,n_L$, derive the total number of trainable parameters in a fully connected neural network.
Answer: In a fully-connected network, each layer $i$ (except the input layer) has a weight matrix connecting all $n_{i-1}$ neurons from the previous layer to the $n_i$ neurons of this layer, plus $n_i$ bias terms. Therefore, the number of parameters in layer $i$ is $n_{i-1}\cdot n_i$ (weights) $+,n_i$ (biases) . Summing over all layers from $1$ to $L$ gives the total number of parameters:

\[
\text{Total params} \;=\; \sum_{i=1}^{L} \Big(n_{i-1}\times n_i + n_i\Big)\,.

\]
For example, a network with architecture $[n_0, n_1, n_2]$ (one hidden layer of size $n_1$) has $n_0n_1 + n_1$ parameters in the first (input-to-hidden) layer and $n_1n_2 + n_2$ in the second (hidden-to-output) layer.

\end{enumerate}

%%%%%%%%%%%%%%%%%%%%%%%%%%%%%%%%%%%%%%%%%%%%%%%%%%%%
\section{Convolutional Neural Networks}

\begin{enumerate}[leftmargin=1.2cm,start=13]

\item[\textbf{13.}]\textbf{(Multiple Choice)}  
Which of the following is \emph{not} an advantage of convolutional networks?  
\begin{enumerate}[label=\alph*)]
\item Local receptive fields  
\item Weight sharing  
\item More parameters than fully-connected layers  
\item Pooling gives translation invariance  
\end{enumerate}

1. True. CNNs use local receptive fields, meaning each neuron in a
convolutional layer connects to only a small region of the input
(spatially).


2. True. CNNs employ weight sharing: the same filter (set of
weights) is applied across different positions of the input, greatly
reducing the number of parameters.


3. True. CNNs have more parameters than
fully-connected networks applied to inputs of the same size, due to
the use of many filters.



4. True. Pooling layers in CNNs help achieve a
degree of translation invariance, by summarizing features over small
neighborhoods.
Convolution and pooling confer two key benefits: far
fewer parameters thanks to local connectivity
and shared filters, and some robustness to translations, e.g. a
feature’s exact location is less critical after pooling . CNNs
leverage these properties to generalize well to image data.

\item[\textbf{14.}]}  

  Zero-padding can preserve spatial dimensions when using $3\times3$
  kernels with stride~1.  Using zero-padding in convolutional layers
  can preserve the spatial size of the input. For example, with a
  $3\times 3$ kernel and stride 1, choosing a padding of $P=1$ on each
  side will keep an input image of size $W \times H$ the same size in
  the output feature map .
True. (In general, the output width for a
  1D convolution is $\frac{W - K + 2P}{S} + 1$. Setting $P = (K-1)/2$
  for stride $S=1$ yields output width $W$. For a $3\times 3$ kernel,
  $(K-1)/2 = 1$, so padding by 1 keeps the output size equal to the
  input size.)

\item[\textbf{15.}]\textbf{(Short Answer)}  
Derive the formula for the output width of a convolutional layer with input width $W$, filter size $K$, stride $S$, and padding $P$.
Answer: The output width $W_{\text{out}}$ is given by
\[
W_{\text{out}} = \frac{W - K + 2P}{S} + 1,
\]

assuming $(W - K + 2P)$ is divisible by $S$. An analogous formula holds for the output height using $H$, and this formula also generalizes to multiple convolutional layers or to the spatial dimensions of feature maps.

\item[\textbf{16.}]\textbf{(Short Answer)}  
A convolutional layer has $C_{\mathrm{in}}$ input channels, $C_{\mathrm{out}}$ filters, and kernel size $K_h \times K_w$.  
Compute the number of trainable parameters (including biases).
Answer: Each filter has $C_{\text{in}} \times K_h \times K_w$ weights, and typically one bias term. With $C_{\text{out}}$ filters in the layer, the total parameter count is
\[
K_h \cdot K_w \cdot C_{\text{in}} + 1) \times C_{\text{out}},

\]
which accounts for all filter weights plus one bias per filter . (For example, a conv layer with $32$ filters of size $3\times 3$ and $3$ input channels has $(3\cdot3\cdot3+1)\times 32 = 896$ parameters .)

\end{enumerate}

%%%%%%%%%%%%%%%%%%%%%%%%%%%%%%%%%%%%%%%%%%%%%%%%%%%%
\section{Recurrent Neural Networks}

\begin{enumerate}[leftmargin=1.2cm,start=17]

\item[\textbf{17.}]\textbf{(Multiple Choice)}  
Which statement about vanilla RNNs is \emph{false}?  
\begin{enumerate}[label=\alph*)]
\item They maintain a hidden state  
\item They use shared weights across time  
\item They can process sequences of arbitrary length  
\item They avoid vanishing gradients  
\end{enumerate}

1. True. RNNs maintain a hidden state vector that is updated at each time
step, allowing the network to retain information from previous
inputs.


2. True. RNNs use the same weight matrices (shared weights) at every
time step of the sequence, instead of having separate weights for each
time step.


3. True. RNNs can, in principle, process input sequences of
arbitrary length (they are not limited to a fixed input size per
se).


4. False. RNNs completely eliminate the vanishing gradient problem, so
they can easily learn long-term dependencies.
(Standard RNNs “unfold”
a single recurrent layer across time steps and share parameters along
the sequence . They maintain an internal memory (hidden state) to
capture temporal dependencies and can handle sequences of varying
length. However, vanilla RNNs do suffer from the vanishing gradient
problem, which makes learning long-term dependencies challenging
. That drawback led to the development of
gated RNN variants.


\item[\textbf{18.}]\textbf{(True/False)}
  
LSTMs mitigate vanishing gradients by introducing gating mechanisms.
True.
Long Short-Term Memory (LSTM) networks were designed to overcome the vanishing gradient issue in RNNs by introducing gating mechanisms that control information flow (e.g. input, forget, and output gates).

LSTMs have a cell state and gates that regulate when to store,
forget, or output information. This architecture enables gradients to
flow better over long time spans, mitigating vanishing gradients and
enabling the network to learn long-term dependencies.

\item[\textbf{19.}]\textbf{(Short Answer)}  
What is Backpropagation Through Time (BPTT), and why is it required for training RNNs?

Answer: BPTT is the adaptation of the standard backpropagation
algorithm for unfolded recurrent neural networks. When an RNN is
“unrolled” over $T$ time steps, it can be viewed as a deep network
with $T$ layers (one per time step). Backpropagation Through Time
entails propagating the error gradients backward through all these
time-step connections (hence “through time”) to compute weight
updates. It is necessary because an RNN’s output at time $t$ depends
on not only the weights at that step but also on the states (and thus
inputs) from previous time steps. BPTT allows the network to assign
credit (or blame) to weights based on sequence-wide outcomes by
accumulating gradients over each time step . Without BPTT, the RNN
would not learn temporal relationships properly, since we must
consider the influence of earlier inputs on later outputs when
adjusting the recurrent weights.


\end{enumerate}

%%%%%%%%%%%%%%%%%%%%%%%%%%%%%%%%%%%%%%%%%%%%%%%%%%%%

\end{document}




