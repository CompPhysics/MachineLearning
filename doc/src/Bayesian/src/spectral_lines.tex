
% Default to the notebook output style

    


% Inherit from the specified cell style.




    
\documentclass{article}

    
    
    \usepackage{graphicx} % Used to insert images
    \usepackage{adjustbox} % Used to constrain images to a maximum size 
    \usepackage{color} % Allow colors to be defined
    \usepackage{enumerate} % Needed for markdown enumerations to work
    \usepackage{geometry} % Used to adjust the document margins
    \usepackage{amsmath} % Equations
    \usepackage{amssymb} % Equations
    \usepackage{eurosym} % defines \euro
    \usepackage[mathletters]{ucs} % Extended unicode (utf-8) support
    \usepackage[utf8x]{inputenc} % Allow utf-8 characters in the tex document
    \usepackage{fancyvrb} % verbatim replacement that allows latex
    \usepackage{grffile} % extends the file name processing of package graphics 
                         % to support a larger range 
    % The hyperref package gives us a pdf with properly built
    % internal navigation ('pdf bookmarks' for the table of contents,
    % internal cross-reference links, web links for URLs, etc.)
    \usepackage{hyperref}
    \usepackage{longtable} % longtable support required by pandoc >1.10
    \usepackage{booktabs}  % table support for pandoc > 1.12.2
    \usepackage{ulem} % ulem is needed to support strikethroughs (\sout)
    

    
    
    \definecolor{orange}{cmyk}{0,0.4,0.8,0.2}
    \definecolor{darkorange}{rgb}{.71,0.21,0.01}
    \definecolor{darkgreen}{rgb}{.12,.54,.11}
    \definecolor{myteal}{rgb}{.26, .44, .56}
    \definecolor{gray}{gray}{0.45}
    \definecolor{lightgray}{gray}{.95}
    \definecolor{mediumgray}{gray}{.8}
    \definecolor{inputbackground}{rgb}{.95, .95, .85}
    \definecolor{outputbackground}{rgb}{.95, .95, .95}
    \definecolor{traceback}{rgb}{1, .95, .95}
    % ansi colors
    \definecolor{red}{rgb}{.6,0,0}
    \definecolor{green}{rgb}{0,.65,0}
    \definecolor{brown}{rgb}{0.6,0.6,0}
    \definecolor{blue}{rgb}{0,.145,.698}
    \definecolor{purple}{rgb}{.698,.145,.698}
    \definecolor{cyan}{rgb}{0,.698,.698}
    \definecolor{lightgray}{gray}{0.5}
    
    % bright ansi colors
    \definecolor{darkgray}{gray}{0.25}
    \definecolor{lightred}{rgb}{1.0,0.39,0.28}
    \definecolor{lightgreen}{rgb}{0.48,0.99,0.0}
    \definecolor{lightblue}{rgb}{0.53,0.81,0.92}
    \definecolor{lightpurple}{rgb}{0.87,0.63,0.87}
    \definecolor{lightcyan}{rgb}{0.5,1.0,0.83}
    
    % commands and environments needed by pandoc snippets
    % extracted from the output of `pandoc -s`
    \providecommand{\tightlist}{%
      \setlength{\itemsep}{0pt}\setlength{\parskip}{0pt}}
    \DefineVerbatimEnvironment{Highlighting}{Verbatim}{commandchars=\\\{\}}
    % Add ',fontsize=\small' for more characters per line
    \newenvironment{Shaded}{}{}
    \newcommand{\KeywordTok}[1]{\textcolor[rgb]{0.00,0.44,0.13}{\textbf{{#1}}}}
    \newcommand{\DataTypeTok}[1]{\textcolor[rgb]{0.56,0.13,0.00}{{#1}}}
    \newcommand{\DecValTok}[1]{\textcolor[rgb]{0.25,0.63,0.44}{{#1}}}
    \newcommand{\BaseNTok}[1]{\textcolor[rgb]{0.25,0.63,0.44}{{#1}}}
    \newcommand{\FloatTok}[1]{\textcolor[rgb]{0.25,0.63,0.44}{{#1}}}
    \newcommand{\CharTok}[1]{\textcolor[rgb]{0.25,0.44,0.63}{{#1}}}
    \newcommand{\StringTok}[1]{\textcolor[rgb]{0.25,0.44,0.63}{{#1}}}
    \newcommand{\CommentTok}[1]{\textcolor[rgb]{0.38,0.63,0.69}{\textit{{#1}}}}
    \newcommand{\OtherTok}[1]{\textcolor[rgb]{0.00,0.44,0.13}{{#1}}}
    \newcommand{\AlertTok}[1]{\textcolor[rgb]{1.00,0.00,0.00}{\textbf{{#1}}}}
    \newcommand{\FunctionTok}[1]{\textcolor[rgb]{0.02,0.16,0.49}{{#1}}}
    \newcommand{\RegionMarkerTok}[1]{{#1}}
    \newcommand{\ErrorTok}[1]{\textcolor[rgb]{1.00,0.00,0.00}{\textbf{{#1}}}}
    \newcommand{\NormalTok}[1]{{#1}}
    
    % Additional commands for more recent versions of Pandoc
    \newcommand{\ConstantTok}[1]{\textcolor[rgb]{0.53,0.00,0.00}{{#1}}}
    \newcommand{\SpecialCharTok}[1]{\textcolor[rgb]{0.25,0.44,0.63}{{#1}}}
    \newcommand{\VerbatimStringTok}[1]{\textcolor[rgb]{0.25,0.44,0.63}{{#1}}}
    \newcommand{\SpecialStringTok}[1]{\textcolor[rgb]{0.73,0.40,0.53}{{#1}}}
    \newcommand{\ImportTok}[1]{{#1}}
    \newcommand{\DocumentationTok}[1]{\textcolor[rgb]{0.73,0.13,0.13}{\textit{{#1}}}}
    \newcommand{\AnnotationTok}[1]{\textcolor[rgb]{0.38,0.63,0.69}{\textbf{\textit{{#1}}}}}
    \newcommand{\CommentVarTok}[1]{\textcolor[rgb]{0.38,0.63,0.69}{\textbf{\textit{{#1}}}}}
    \newcommand{\VariableTok}[1]{\textcolor[rgb]{0.10,0.09,0.49}{{#1}}}
    \newcommand{\ControlFlowTok}[1]{\textcolor[rgb]{0.00,0.44,0.13}{\textbf{{#1}}}}
    \newcommand{\OperatorTok}[1]{\textcolor[rgb]{0.40,0.40,0.40}{{#1}}}
    \newcommand{\BuiltInTok}[1]{{#1}}
    \newcommand{\ExtensionTok}[1]{{#1}}
    \newcommand{\PreprocessorTok}[1]{\textcolor[rgb]{0.74,0.48,0.00}{{#1}}}
    \newcommand{\AttributeTok}[1]{\textcolor[rgb]{0.49,0.56,0.16}{{#1}}}
    \newcommand{\InformationTok}[1]{\textcolor[rgb]{0.38,0.63,0.69}{\textbf{\textit{{#1}}}}}
    \newcommand{\WarningTok}[1]{\textcolor[rgb]{0.38,0.63,0.69}{\textbf{\textit{{#1}}}}}
    
    
    % Define a nice break command that doesn't care if a line doesn't already
    % exist.
    \def\br{\hspace*{\fill} \\* }
    % Math Jax compatability definitions
    \def\gt{>}
    \def\lt{<}
    % Document parameters
    \title{A spectral line problem}
    
    \author{Christian Forss\'en, Department of Physics, Chalmers}
    

    % Pygments definitions
    
\makeatletter
\def\PY@reset{\let\PY@it=\relax \let\PY@bf=\relax%
    \let\PY@ul=\relax \let\PY@tc=\relax%
    \let\PY@bc=\relax \let\PY@ff=\relax}
\def\PY@tok#1{\csname PY@tok@#1\endcsname}
\def\PY@toks#1+{\ifx\relax#1\empty\else%
    \PY@tok{#1}\expandafter\PY@toks\fi}
\def\PY@do#1{\PY@bc{\PY@tc{\PY@ul{%
    \PY@it{\PY@bf{\PY@ff{#1}}}}}}}
\def\PY#1#2{\PY@reset\PY@toks#1+\relax+\PY@do{#2}}

\expandafter\def\csname PY@tok@s\endcsname{\def\PY@tc##1{\textcolor[rgb]{0.73,0.13,0.13}{##1}}}
\expandafter\def\csname PY@tok@kd\endcsname{\let\PY@bf=\textbf\def\PY@tc##1{\textcolor[rgb]{0.00,0.50,0.00}{##1}}}
\expandafter\def\csname PY@tok@c\endcsname{\let\PY@it=\textit\def\PY@tc##1{\textcolor[rgb]{0.25,0.50,0.50}{##1}}}
\expandafter\def\csname PY@tok@bp\endcsname{\def\PY@tc##1{\textcolor[rgb]{0.00,0.50,0.00}{##1}}}
\expandafter\def\csname PY@tok@ss\endcsname{\def\PY@tc##1{\textcolor[rgb]{0.10,0.09,0.49}{##1}}}
\expandafter\def\csname PY@tok@nd\endcsname{\def\PY@tc##1{\textcolor[rgb]{0.67,0.13,1.00}{##1}}}
\expandafter\def\csname PY@tok@w\endcsname{\def\PY@tc##1{\textcolor[rgb]{0.73,0.73,0.73}{##1}}}
\expandafter\def\csname PY@tok@s1\endcsname{\def\PY@tc##1{\textcolor[rgb]{0.73,0.13,0.13}{##1}}}
\expandafter\def\csname PY@tok@gp\endcsname{\let\PY@bf=\textbf\def\PY@tc##1{\textcolor[rgb]{0.00,0.00,0.50}{##1}}}
\expandafter\def\csname PY@tok@m\endcsname{\def\PY@tc##1{\textcolor[rgb]{0.40,0.40,0.40}{##1}}}
\expandafter\def\csname PY@tok@il\endcsname{\def\PY@tc##1{\textcolor[rgb]{0.40,0.40,0.40}{##1}}}
\expandafter\def\csname PY@tok@gd\endcsname{\def\PY@tc##1{\textcolor[rgb]{0.63,0.00,0.00}{##1}}}
\expandafter\def\csname PY@tok@k\endcsname{\let\PY@bf=\textbf\def\PY@tc##1{\textcolor[rgb]{0.00,0.50,0.00}{##1}}}
\expandafter\def\csname PY@tok@sc\endcsname{\def\PY@tc##1{\textcolor[rgb]{0.73,0.13,0.13}{##1}}}
\expandafter\def\csname PY@tok@nc\endcsname{\let\PY@bf=\textbf\def\PY@tc##1{\textcolor[rgb]{0.00,0.00,1.00}{##1}}}
\expandafter\def\csname PY@tok@kp\endcsname{\def\PY@tc##1{\textcolor[rgb]{0.00,0.50,0.00}{##1}}}
\expandafter\def\csname PY@tok@cpf\endcsname{\let\PY@it=\textit\def\PY@tc##1{\textcolor[rgb]{0.25,0.50,0.50}{##1}}}
\expandafter\def\csname PY@tok@gt\endcsname{\def\PY@tc##1{\textcolor[rgb]{0.00,0.27,0.87}{##1}}}
\expandafter\def\csname PY@tok@sd\endcsname{\let\PY@it=\textit\def\PY@tc##1{\textcolor[rgb]{0.73,0.13,0.13}{##1}}}
\expandafter\def\csname PY@tok@vc\endcsname{\def\PY@tc##1{\textcolor[rgb]{0.10,0.09,0.49}{##1}}}
\expandafter\def\csname PY@tok@na\endcsname{\def\PY@tc##1{\textcolor[rgb]{0.49,0.56,0.16}{##1}}}
\expandafter\def\csname PY@tok@sh\endcsname{\def\PY@tc##1{\textcolor[rgb]{0.73,0.13,0.13}{##1}}}
\expandafter\def\csname PY@tok@ch\endcsname{\let\PY@it=\textit\def\PY@tc##1{\textcolor[rgb]{0.25,0.50,0.50}{##1}}}
\expandafter\def\csname PY@tok@no\endcsname{\def\PY@tc##1{\textcolor[rgb]{0.53,0.00,0.00}{##1}}}
\expandafter\def\csname PY@tok@gs\endcsname{\let\PY@bf=\textbf}
\expandafter\def\csname PY@tok@mo\endcsname{\def\PY@tc##1{\textcolor[rgb]{0.40,0.40,0.40}{##1}}}
\expandafter\def\csname PY@tok@si\endcsname{\let\PY@bf=\textbf\def\PY@tc##1{\textcolor[rgb]{0.73,0.40,0.53}{##1}}}
\expandafter\def\csname PY@tok@gh\endcsname{\let\PY@bf=\textbf\def\PY@tc##1{\textcolor[rgb]{0.00,0.00,0.50}{##1}}}
\expandafter\def\csname PY@tok@mb\endcsname{\def\PY@tc##1{\textcolor[rgb]{0.40,0.40,0.40}{##1}}}
\expandafter\def\csname PY@tok@go\endcsname{\def\PY@tc##1{\textcolor[rgb]{0.53,0.53,0.53}{##1}}}
\expandafter\def\csname PY@tok@s2\endcsname{\def\PY@tc##1{\textcolor[rgb]{0.73,0.13,0.13}{##1}}}
\expandafter\def\csname PY@tok@nf\endcsname{\def\PY@tc##1{\textcolor[rgb]{0.00,0.00,1.00}{##1}}}
\expandafter\def\csname PY@tok@sx\endcsname{\def\PY@tc##1{\textcolor[rgb]{0.00,0.50,0.00}{##1}}}
\expandafter\def\csname PY@tok@nv\endcsname{\def\PY@tc##1{\textcolor[rgb]{0.10,0.09,0.49}{##1}}}
\expandafter\def\csname PY@tok@kt\endcsname{\def\PY@tc##1{\textcolor[rgb]{0.69,0.00,0.25}{##1}}}
\expandafter\def\csname PY@tok@mh\endcsname{\def\PY@tc##1{\textcolor[rgb]{0.40,0.40,0.40}{##1}}}
\expandafter\def\csname PY@tok@ow\endcsname{\let\PY@bf=\textbf\def\PY@tc##1{\textcolor[rgb]{0.67,0.13,1.00}{##1}}}
\expandafter\def\csname PY@tok@gr\endcsname{\def\PY@tc##1{\textcolor[rgb]{1.00,0.00,0.00}{##1}}}
\expandafter\def\csname PY@tok@kc\endcsname{\let\PY@bf=\textbf\def\PY@tc##1{\textcolor[rgb]{0.00,0.50,0.00}{##1}}}
\expandafter\def\csname PY@tok@cm\endcsname{\let\PY@it=\textit\def\PY@tc##1{\textcolor[rgb]{0.25,0.50,0.50}{##1}}}
\expandafter\def\csname PY@tok@kr\endcsname{\let\PY@bf=\textbf\def\PY@tc##1{\textcolor[rgb]{0.00,0.50,0.00}{##1}}}
\expandafter\def\csname PY@tok@se\endcsname{\let\PY@bf=\textbf\def\PY@tc##1{\textcolor[rgb]{0.73,0.40,0.13}{##1}}}
\expandafter\def\csname PY@tok@ne\endcsname{\let\PY@bf=\textbf\def\PY@tc##1{\textcolor[rgb]{0.82,0.25,0.23}{##1}}}
\expandafter\def\csname PY@tok@gu\endcsname{\let\PY@bf=\textbf\def\PY@tc##1{\textcolor[rgb]{0.50,0.00,0.50}{##1}}}
\expandafter\def\csname PY@tok@sb\endcsname{\def\PY@tc##1{\textcolor[rgb]{0.73,0.13,0.13}{##1}}}
\expandafter\def\csname PY@tok@vg\endcsname{\def\PY@tc##1{\textcolor[rgb]{0.10,0.09,0.49}{##1}}}
\expandafter\def\csname PY@tok@nt\endcsname{\let\PY@bf=\textbf\def\PY@tc##1{\textcolor[rgb]{0.00,0.50,0.00}{##1}}}
\expandafter\def\csname PY@tok@cs\endcsname{\let\PY@it=\textit\def\PY@tc##1{\textcolor[rgb]{0.25,0.50,0.50}{##1}}}
\expandafter\def\csname PY@tok@cp\endcsname{\def\PY@tc##1{\textcolor[rgb]{0.74,0.48,0.00}{##1}}}
\expandafter\def\csname PY@tok@mi\endcsname{\def\PY@tc##1{\textcolor[rgb]{0.40,0.40,0.40}{##1}}}
\expandafter\def\csname PY@tok@kn\endcsname{\let\PY@bf=\textbf\def\PY@tc##1{\textcolor[rgb]{0.00,0.50,0.00}{##1}}}
\expandafter\def\csname PY@tok@o\endcsname{\def\PY@tc##1{\textcolor[rgb]{0.40,0.40,0.40}{##1}}}
\expandafter\def\csname PY@tok@c1\endcsname{\let\PY@it=\textit\def\PY@tc##1{\textcolor[rgb]{0.25,0.50,0.50}{##1}}}
\expandafter\def\csname PY@tok@mf\endcsname{\def\PY@tc##1{\textcolor[rgb]{0.40,0.40,0.40}{##1}}}
\expandafter\def\csname PY@tok@vi\endcsname{\def\PY@tc##1{\textcolor[rgb]{0.10,0.09,0.49}{##1}}}
\expandafter\def\csname PY@tok@sr\endcsname{\def\PY@tc##1{\textcolor[rgb]{0.73,0.40,0.53}{##1}}}
\expandafter\def\csname PY@tok@nl\endcsname{\def\PY@tc##1{\textcolor[rgb]{0.63,0.63,0.00}{##1}}}
\expandafter\def\csname PY@tok@nb\endcsname{\def\PY@tc##1{\textcolor[rgb]{0.00,0.50,0.00}{##1}}}
\expandafter\def\csname PY@tok@nn\endcsname{\let\PY@bf=\textbf\def\PY@tc##1{\textcolor[rgb]{0.00,0.00,1.00}{##1}}}
\expandafter\def\csname PY@tok@gi\endcsname{\def\PY@tc##1{\textcolor[rgb]{0.00,0.63,0.00}{##1}}}
\expandafter\def\csname PY@tok@ni\endcsname{\let\PY@bf=\textbf\def\PY@tc##1{\textcolor[rgb]{0.60,0.60,0.60}{##1}}}
\expandafter\def\csname PY@tok@err\endcsname{\def\PY@bc##1{\setlength{\fboxsep}{0pt}\fcolorbox[rgb]{1.00,0.00,0.00}{1,1,1}{\strut ##1}}}
\expandafter\def\csname PY@tok@ge\endcsname{\let\PY@it=\textit}

\def\PYZbs{\char`\\}
\def\PYZus{\char`\_}
\def\PYZob{\char`\{}
\def\PYZcb{\char`\}}
\def\PYZca{\char`\^}
\def\PYZam{\char`\&}
\def\PYZlt{\char`\<}
\def\PYZgt{\char`\>}
\def\PYZsh{\char`\#}
\def\PYZpc{\char`\%}
\def\PYZdl{\char`\$}
\def\PYZhy{\char`\-}
\def\PYZsq{\char`\'}
\def\PYZdq{\char`\"}
\def\PYZti{\char`\~}
% for compatibility with earlier versions
\def\PYZat{@}
\def\PYZlb{[}
\def\PYZrb{]}
\makeatother


    % Exact colors from NB
    \definecolor{incolor}{rgb}{0.0, 0.0, 0.5}
    \definecolor{outcolor}{rgb}{0.545, 0.0, 0.0}



    
    % Prevent overflowing lines due to hard-to-break entities
    \sloppy 
    % Setup hyperref package
    \hypersetup{
      breaklinks=true,  % so long urls are correctly broken across lines
      colorlinks=true,
      urlcolor=blue,
      linkcolor=darkorange,
      citecolor=darkgreen,
      }
    % Slightly bigger margins than the latex defaults
    
    \geometry{verbose,tmargin=1in,bmargin=1in,lmargin=1in,rmargin=1in}
    
    

    \begin{document}
    
    
    \maketitle
    
    

    
    See e.g.~Section 4.2 in Sivia for a similar problem formulation. In
short, we have data from a spectroscopy experiment that supposedly shows
a number of spectral lines. The ideal spectrum ca be expressed as

\[ G(x) = \sum_{j=1}^M A_j f(x,x_j),\]

where \(A_j\) is the amplitude of the \(j\)th line, and \(x_j\)
represents its position. If all the spectral lines were Gaussians of
width \(W\), for example, then

\[ f(x,x_j) = \exp \left[ - \frac{(x-x_j)^2}{2 W^2} \right]\]

    includes a background signal \(\{ B_k\}\). We use the label `\(k\)' to
enumerate the positions \(\{x_k\}\).

The ideal spectrum according to our model is therefore

\[ F_k \equiv F(x_k) = G(x_k) + B(x_k).\]

    The experimental data is denoted \(\{ D_k\}\). This data also includes
measurement errors \(\{ \varepsilon_k\}\) that are assumed to be
independent and identically distributed (IID) normal with some variance
\(\sigma_k\). The measured data is then related to the ideal spectrum by

\[ D_k \equiv D(x_k) = G(x_k) + B(x_k) + \varepsilon(x_k).\]

A simulated data set is shown in the figure. 
    \begin{center}
    \adjustimage{max size={0.8\linewidth}{0.8\paperheight}}{spectral_lines_files/spectral_lines_22_0.png}
    \end{center}
    { \hspace*{\fill} \\}


\paragraph{Problem task:}
The task is to infer the positions (\(x_j\)) and amplitudes (\(A_j\)) of
the spectral lines from the experimental data.

\subsection*{Known information}
%
Let us use a model that assumes two spectral lines. In addition we
assert the following known information: 
%
\begin{enumerate}
\item a known, constant background ($B$). 
\item a known, natural width
(W) of the spectral lines (the same for both). 
\item a known variance ($\sigma_k$) for the IID
normal experimental errors.
\item a known and relevant interval $[x_\mathrm{min},
x_\mathrm{max}]$ in position space.
\end{enumerate}
%
The following numerical data is used in this example:

    \begin{Verbatim}[commandchars=\\\{\}]
Constant background:                      B\_k = B = 0.2
Natural width of spectral lines:                W = 0.1
Variance for IID normal exp errors:             s = 0.05
Relevant range in position space:    [xmin, xmax] = [0.0, 2.0]
    \end{Verbatim}


\subsection*{Suggested solution strategy---known number of spectral lines}\label{solution-strategy}

Assuming that our model \(M\)  describes a spectrum with two spectral
lines, it will have five model parameters. We denote them by the vector
\(\vec{\alpha}\). These are the amplitudes and positions of the two
lines (the width is assumed to be known), and the
constant background. We order them as follows:

\[ \vec{\alpha} = (A_0, x_0, A_1, x_1, B).\]

The background strength ($B$) is a \emph{nuisance parameter} in the sense that
we're not really interested in its value, we just need to marginalize
over it.

    We start with wirting down Bayes' theorem

\[ p(\alpha | \{D_k\}, I) = \frac{p(\{D_k\} | \vec{\alpha}, I) p(\vec{\alpha}|I)}{p(\{D_k\} | I)},\]

where the prior information includes expectations of the number of peaks
(two in this case, their natural width, the experimental errors, etc).
In the following, we will ignore the denominator (the \emph{evidence} or
\emph{marginal likelihood}) as it just constitutes a normalization
factor to the posterior pdf (the left hand side), which is the quantity
that we are interested in.

    It is your task, however, to make reasonable assumptions for the prior
(\emph{hint:} uniform ones are easy to work with). We can assume that
the available information implies that we expect the amplitude to be a
positive quantity, and that it is expeted to be of order 1 (i.e.~not
10). The peak positions can be assumed to be in the {[}xmin, xmax{]}
range of the data, and the background signal is supposedly at least an
order of magnitude smaller than the peak amplitudes.

    Concerning the \emph{likelihood} (the first factor in the nominator),
our information on the IID errors of the experiment leads to the
least-squares likelihood (see Sivia, Sec. 3.5):

\[ p(\{D_k\} | \vec{\alpha}, I) \propto \exp(-\chi^2/2),\]

where the chi-squared function is defined by the sum of squared
\emph{residuals}

\[ \chi^2 = \sum_{k=1}^N \left( \frac{F_k - D_k}{\sigma_k} \right)^2.\]

Note that the normalization factor of the likelihood (involving the
product of terms \(\sqrt{2\pi}\sigma_k\)) has been omitted as it does
not depend on the parameters of our model.

\subsection*{Simulated data}

It can be
loaded from the file \texttt{data\_spectral\_lines.txt}. The columns
of this file correspond to the data triples: $(x_k, D_k, \sigma_k)$, as
also listed below.
    
    \begin{Verbatim}[commandchars=\\\{\}]
\#   x\_k      D\_k   s\_k
\#  ----     ----   ---
 0.0000   0.2046   0.1
 0.0202   0.2546   0.1
 0.0404   0.1027   0.1
 0.0606   0.1307   0.1
 0.0808   0.0852   0.1
 0.1010   0.3205   0.1
 0.1212   0.2864   0.1
 0.1414   0.3102   0.1
 0.1616   0.2397   0.1
 0.1818   0.2488   0.1
 0.2020   0.1408   0.1
 0.2222   0.2958   0.1
 0.2424   0.1438   0.1
 0.2626   0.1668   0.1
 0.2828   0.1811   0.1
 0.3030   0.1604   0.1
 0.3232   0.2430   0.1
 0.3434   0.1885   0.1
 0.3636   0.1967   0.1
 0.3838   0.1896   0.1
 0.4040   0.2673   0.1
 0.4242   0.1697   0.1
 0.4444   0.1913   0.1
 0.4646   0.2212   0.1
 0.4848   0.1178   0.1
 0.5051   0.1760   0.1
 0.5253   0.2272   0.1
 0.5455   0.2592   0.1
 0.5657   0.2100   0.1
 0.5859   0.1700   0.1
 0.6061   0.2788   0.1
 0.6263   0.2007   0.1
 0.6465   0.2217   0.1
 0.6667   0.2420   0.1
 0.6869   0.2485   0.1
 0.7071   0.2507   0.1
 0.7273   0.3364   0.1
 0.7475   0.2984   0.1
 0.7677   0.3841   0.1
 0.7879   0.4480   0.1
 0.8081   0.4845   0.1
 0.8283   0.5976   0.1
 0.8485   0.6578   0.1
 0.8687   0.7028   0.1
 0.8889   0.7288   0.1
 0.9091   0.7914   0.1
 0.9293   0.8692   0.1
 0.9495   0.8534   0.1
 0.9697   0.6716   0.1
 0.9899   0.9239   0.1
 1.0101   0.8303   0.1
 1.0303   0.9112   0.1
 1.0505   1.0990   0.1
 1.0707   1.1252   0.1
 1.0909   1.3663   0.1
 1.1111   1.4013   0.1
 1.1313   1.3731   0.1
 1.1515   1.3388   0.1
 1.1717   1.4010   0.1
 1.1919   1.3674   0.1
 1.2121   1.1981   0.1
 1.2323   1.1509   0.1
 1.2525   0.9372   0.1
 1.2727   0.8055   0.1
 1.2929   0.6170   0.1
 1.3131   0.5362   0.1
 1.3333   0.4122   0.1
 1.3535   0.2570   0.1
 1.3737   0.4218   0.1
 1.3939   0.2539   0.1
 1.4141   0.2424   0.1
 1.4343   0.2876   0.1
 1.4545   0.2171   0.1
 1.4747   0.2169   0.1
 1.4949   0.2246   0.1
 1.5152   0.1440   0.1
 1.5354   0.2770   0.1
 1.5556   0.2183   0.1
 1.5758   0.1696   0.1
 1.5960   0.2466   0.1
 1.6162   0.2178   0.1
 1.6364   0.1524   0.1
 1.6566   0.2076   0.1
 1.6768   0.2000   0.1
 1.6970   0.2168   0.1
 1.7172   0.2187   0.1
 1.7374   0.2357   0.1
 1.7576   0.1400   0.1
 1.7778   0.2001   0.1
 1.7980   0.1539   0.1
 1.8182   0.2248   0.1
 1.8384   0.1334   0.1
 1.8586   0.0905   0.1
 1.8788   0.1782   0.1
 1.8990   0.3273   0.1
 1.9192   0.2575   0.1
 1.9394   0.2042   0.1
 1.9596   0.3243   0.1
 1.9798   0.1903   0.1
 2.0000   0.1907   0.1
    \end{Verbatim}

    \subsection*{Possible extension 1:}\label{extension-1}

Assume that the detectors are characterized by a finite resolution that
is described by a resolution function so that the ideal data is related
to the ideal spectrum
\[ \tilde{D}_k = \int G(x) R(x_k - x) dx + B(x_k) + \sigma(x_k),\] where
we have implicitly assumed that the resolution function \(R(x)\) does
not vary with position in writing the blurring process as a convolution
integral.

\subsection*{Possible extension 2:}\label{extension-2}

A possible extension of this problem is to compare the model evidences
for two competing hypothesis of the number of spectral lines in the
data. E.g. compute the evidence for a model \(M_1\) that corresponds to
the hypothesis that the experimental data can be explained with \emph{a
single} spectral line, and model \(M_2\) that assumes \emph{two}
spectral lines.

    % Add a bibliography block to the postdoc
    
    
    
    \end{document}
